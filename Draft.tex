\documentclass[useAMS,usenatbib,referee,12pt]{article}
\usepackage[margin=1.0in]{geometry}
\usepackage{amsmath}
\usepackage{amssymb}
\usepackage{amsfonts}
\usepackage{parskip}
\usepackage[round]{natbib}
\usepackage{color}
\usepackage[dvipsnames]{xcolor}
\usepackage{caption}
\usepackage{tabularx, booktabs}
\usepackage{float}
\usepackage{adjustbox}
\usepackage[toc,page]{appendix}
\usepackage{enumitem}
\usepackage{multirow}
\usepackage{setspace}
\doublespacing

\newif\ifdic   % Type \ifdic at the beginning of any optional section... use \fi to denote the end of the optional section
  % See: http://tex.stackexchange.com/questions/33576/conditional-typesetting-build
%\dictrue         % comment out if I want to hide DIC and model fit paragraphs

\newcommand{\adam}[1]{{\color{blue} ADAM: #1}}
\newcommand{\jarad}[1]{{\color{Orange} JARAD: #1}}
\newenvironment{indpar}[1]%
     {\begin{list}{}%
             {\setlength{\leftmargin}{#1}}%
             \item[]%
     }
     {\end{list}}

\newcommand{\vn}{\textbf{n}}
\newcommand{\vp}{\textbf{p}}
\newcommand{\vX}{\textbf{X}}
\newcommand{\vZ}{\textbf{Z}}
\newcommand{\vbeta}{\boldsymbol{\beta}}
\newcommand{\vxi}{\boldsymbol{\xi}}

\newcommand{\Exp}{\mbox{Exp}}
\newcommand{\Ga}{\mbox{Ga}}
\newcommand{\We}{\mbox{We}}
\newcommand{\LN}{\mbox{LN}}
\newcommand{\Po}{\mbox{Po}}
\newcommand{\Mult}{\mbox{Mult}}

\newcommand{\pdet}{p^{(det)}}
\newcommand{\ind}{\stackrel{ind}{\sim}}
\newcommand{\Fm}{F_T^{(M)}}

\title{Hierarchical model assessing the impacts of time to detection distribution assumptions on detection probability estimation}

\author{Adam Martin-Schwarze, Jarad Niemi, and Philip Dixon}

\begin{document}

\maketitle
\newpage

%\tableofcontents 

\begin{abstract}

Abundance estimates from animal point-count surveys require accurate estimates of detection probabilities.  
The standard model for estimating detection from removal-sampled point-count surveys assumes that organisms at a survey site are detected at a constant rate; however, this assumption is often not justified.  
%Detection rates can be influenced by organismal behaviors, survey methodology, and observer effort.  
%Failure to account for non-constant detection leads to biased estimates of detection and therefore of abundance.  
We consider a class of N-mixture model that allows for detection heterogeneity over time through a flexibly defined time-to-detection distribution (TTDD) and allows for fixed and random effects for both abundance and detection.
%We consider a class of N-mixture models that allow for detection heterogeniety through a mixture component, flexible two-parameter time-to-detection distributions (TTDDs), and allows for fixed and random effects for both abundance and detection.
%Whereas, under the standard approach, detection probability is modeled by dividing the observation period into equal-duration intervals, we instead model the detection {\it rate} in continuous time via a time-to-detection distribution (TTDD) embedded within a hierarchical N-mixture framework.  
Our model is thus a combination of survival time-to-event analysis with unknown-N, unknown-p abundance estimation.  
We specifically explore two-parameter families of TTDDs (e.g. gamma) that can additionally include a mixture component to model increased probability of detection in the initial observation period.
We find that modeling a TTDD by using a mixture component is necessary when data have a chance of arising from a distribution of this nature.
In addition, two-parameter distributions can outperform exponential-based models both when the truth is exponential or non-exponential.
Finally, we analyze an Overbird data set from the Chippewa National Forest using mixed effect models for both abundance and detection.
We demonstrate that the effects of explanatory variables on abundance and detection are consistent across mixture TTDDs but that flexible TTDDs result in lower estimated probabilities of detection and therefore higher estimates of abundance. 
%We find that modeling a TTDD by using a mixture component for increased probability of detection in the initial observation period is necessary when data have a chance of arising from distributions of this nature.
%In addition, the flexibility offered by two-parameter distributions, e.g. gamma, can outperform exponential based models when the truth is exponential or non-exponential.
%Finally, we analyze an Overbird data set from the Chippewa National Forest using mixed effect models for both abundance and detection and demonstrate that the effects of explanatory variables on abundance and detection are consistent across mixture TTDDs and that flexible TTDDs result in lower estimated probabilities of detection and therefore higher estimates of abundance. 
% We apply this model to Ovenbird counts from the Chippewa National Forest and to datasets simulated under different time-to-detection patterns.  
% Models assuming constant detection rates produce biased estimates of detection when true detection rates vary with time, whereas models allowing for variable detection (assuming gamma, Weibull, or lognormal distributed times to detection) produce less biased estimates of the detection probability and nominal credible interval coverage.  
% Models ignoring detection heterogeneity across subgroups yield biased estimates of detection when such heterogeneity exists, whereas models accounting for detection heterogeneity (modeled as a mixture) return reasonable coverage rates and can outperform heterogeneity-ignorant models even when there is no heterogeneity.

{\bf Keywords:} abundance; availability; N-mixture model; point counts; removal sampling; survival analysis; Bayesian

\end{abstract}

\newpage
\section{Introduction}\label{sec:intro}

Abundance estimates from animal point-count surveys require accurate estimates of detection probabilities.  
Removal sampling, where individuals are solely counted on their first capture, provides one established methodology for estimating detection probabilities \citep{Farnsworth2002}.
% cost-effective in that it requires neither multiple observers nor multiple site visits.
%Removal sampling is a species abundance surveying methodology in which observers capture and physically remove animals at a survey site over a series of trapping sessions.
%Assuming that individual capture rates remain constant over all trapping sessions, it is possible to estimate the proportion of individuals still not captured from the pattern of captures over time .  
%\citet{Farnsworth2002} adapted removal sampling to point-count surveys, suggesting that if: (i) observers record only the first time-to-detection for each animal, and (ii) times to first detection are interval-censored, then data will be similar in for to traditional removal sampled data.
%In the analysis of such data, the observation period is subdivided into unit intervals of equal duration, each of which is equivalent to a trapping session.
%Farnsworth's removal assumptions:\\
%(i) Closed population during survey\\
%(ii) no double-counting\\
%(iii) easy-to-detect group is entirely counted during first interval\\
%(iv) TTDD for hard-to-detect group is correct after the end of the observation period\\
%(v) if using limited radius counts, they are accurate
% Johnson2008: all birds present are available for detection
A typical assumption in removal sampling is a constant detection rate throughout the observation period, but this assumption is often unjustified \citep{Alldredge2007}. %\adam{Others, like Mantyniemi.}  
In particular, animal behaviors such as intermittent singing in birds and frogs or diving in whales \citep{Scott2005, Diefenbach2007, Reidy2011}, differences in behavior across subgroups of animals \citep{Farnsworth2005}, observer impacts on animal behaviors \citep{McSheaRappole1997, Rosenstock2002, Alldredge2007}, and variations in observer effort, e.g. lack of settling in period \citep{LeeMarsden2008, Johnson2008}, can all lead to time-varying rates of detection.
%In particular, animal behaviors, e.g. intermittent singing in birds and frogswhales \citep{Scott2005, Diefenbach2007, Reidy2011}, observer behaviors, e.g. pre-counting and disturbing species \citep{McSheaRappole1997, Rosenstock2002, Alldredge2007, LeeMarsden2008, Johnson2008}, and survey methodology, e.g. lack of settling in period \jarad{what else were you thinking? ref?}, can all lead to detection heterogeneity \citep{Farnsworth2005}.
%Survey methodology itself can affect recording of detection times.  
% Observers may become aware of animals during their settling-in period, resulting in elevated counts early during in the survey period \citep{LeeMarsden2008}.  
% Observers may also become distracted as the survey progresses \citep{Johnson2008}.  
%In the presence of detection heterogeneity, easily detected animals will be removed first, leaving only harder to detect animals, thus resulting in a marginal detection rate that varies over time .

In this manuscript, we develop a model for scenarios where detection rates are not constant over time. 
We consider the first time-to-detection as is done in survival analysis, defining a continuous random variable $T$ for each animal's time to first detection with a probability density function (pdf) $f_T(t)$ and cumulative distribution function (cdf) $F_T(t)$.  
We refer to the distribution of $T$ as a time-to-detection distribution (TTDD).  
One common strategy to deal with data that do not fit a constant-detection assumption is to model increased detection probability in the initial observation period via a mixture component \citep{Farnsworth2002, Farnsworth2005, Alldredge2007, EffordDawson2009, Etterson2009, Reidy2011}, although this is not yet the standard \citep{Solymos2013, Amundson2014, Reidy2016}. 
We consider the choice of whether to include a mixture component in conjunction with TTDDs with non-constant rates.

Unlike most survival analyses, the number of individuals $N$ present at a survey is unknown and may be the primary quantity of interest.  
We embed the TTDD in a hierarchical framework for multinomial counts using an N-mixture model \citep{Wyatt2002, Royle2004NMixture}.  
For our purposes, the N-mixture framework provides three clear benefits: 1) its multinomial data framework accords with the interval-censored data collection that is customary in point-count surveys \citep{Ralph1995}, 2) the hierarchical structure readily lends itself to including abundance- and detection-related covariates and random effects \citep{Dorazio2005, Etterson2009, Amundson2014}, and 3) for a Bayesian analysis, we can sample the posterior joint distribution of N-mixture parameters straight-forwardly using Markov chain Monte Carlo (MCMC).  
The N-mixture framework models abundance as a latent variable with a Poisson or other discrete distribution and independently models detection probabilities.  
Several previous studies have employed the N-mixture framework to analyze removal sampled point-count data while assuming constant detection rates \citep{Royle2004Generalized, Dorazio2005, Etterson2009, Solymos2013, Amundson2014}.  



%Application of this methodology generally assumes that organisms at a survey site are detected at a constant rate \citep{Farnsworth2005, Etterson2009, Solymos2013, Amundson2014, Reidy2016}; however, this assumption is often not justified in the field.  
%Failure to account for non-constant detection leads to biased estimates of detection and therefore of abundance.  



Framing a model in terms of time-to-detection leads to two practical differences vis-a-vis constant-detection models.  
First, in order to model covariate and random effects on detection, we perform mixed effects linear regression on the log of the rate parameter as in \citet{Solymos2013}, whereas most existing studies instead construct regression models on the logit of the equal-interval detection probability.  
The latter is not possible when detection rates are not constant.  
Second, because we can obtain interval-specific detection probabilities from the TTDD by partitioning its cdf, we can directly model the data according to their existing interval structure rather than subdividing the observation period into intervals of equal duration.  
Indeed, with a few simplifications, our model fits exact time-to-detection data, whereas existing constant-detection removal models only approximate exact data by subdividing the observation interval into a large number of fine intervals \citep{Reidy2011, Amundson2014}.
% 
% \adam{Based on further lit review, I think I need to insert a paragraph here about what time-varying modeling has been done.  
% In the fisheries world, models of time-varying catchability have been implemented -- see Schnute 1983, Scruton and Gibson (ref: Mantyniemi), and Mantyniemi2015 for examples.  
% Also, Alldredge 2007 implements a complete-history of detection model (technically mark-recapture, not removal) that estimates interval-specific detection probabilities (complete with covariates and detection heterogeneity).  
% However, it cannot be applied in a removal context, because removal data contains only first times to detection, and the Alldredge model requires subsequent observations -- removal models compensate by specifying a TTDD, thus enabling us to extrapolate beyond the observation period.}  

Section \ref{sec:data} provides a description of the interval-censored time-to-detection avian point count data under consideration.
Section \ref{sec:model} introduces an N-mixture model with a generically defined TTDD for estimating abundance from removal-sampled point-count surveys.
In addition, we discuss reasonable default priors and MCMC analysis used to estimate parameters in these models.
Section \ref{sec:sim} provides three simulation studies to assess the impact of TTDD choice on estimated detection probability. 
Section \ref{sec:ovenbirds} analyzes an Ovenbird data set under different TTDDs to determine the impact of this choice on estimated detection probability and therefore estimated abundance.
Finally, section \ref{sec:discuss} discusses 
\adam{The previous sentence ends mid-thought.  Is this for the Discussion section?}


\section{Interval-censored avian point counts}\label{sec:data}

Our analysis is motivated by avian point-count surveys in Chippewa National Forest from 2008-2013 as part of the Minnesota Forest Breeding Bird Project (MNFB) \citep{Hanowski1995}.  
Survey sites were selected from sawtimber red pine stands with no recent logging activity.  
Each stand had up to four sites with sufficient geographical distance between sites to reduce or eliminate overlapping territories.
The data include 65 sites and a total of 381 surveys with site specific variables including site age,  stock density, an indicator of select-/partial-cut logging during the 1990s. 

Single-visit (per year) point-count surveys were conducted by trained observers at each site once annually (weather permitting).  
Fourteen different observers conducted surveys during the study period.  
69\% of surveys in our dataset involved observers in their first year at MNFB.  
Survey durations were 10 minutes, with times to first detection for over 50 species censored into nine intervals: a two-minute interval followed by eight one-minute intervals.  
During each survey, the Julian date, time of day, and temperature were recorded. 

\jarad{Put information about Ovenbird here?}

a total of 947 Ovenbirds counted and a maximum of eight birds counted in any single survey.






\section{Continuous time-to-detection N-mixture models} \label{sec:model}

Before considering interval censoring and explanatory variables, we first present the scenario of exact time to detections with no explanatory variables. 
We then incorporate interval censoring and follow with inclusion of fixed and random effects for abundance and detection. 

\subsection{Exact time to detection}

Suppose that, for each survey $s$ ($s=1,\dots,S$), $N_s$ individuals are present.  
Imagine an observer could remain at the survey location until every individual is detected, marking the time to detection $t_{sb}$ (for bird, $b=1,\dots,N_s$) for each.
Then, assuming detection times for all individuals at a survey are independent, identically distributed according to a common time-to-detection distribution (TTDD), we define $T_{sb}$ as a random variable with cumulative distribution function (cdf) $F_T(t)$ and probability density function (pdf) $f_T(t)$.
However, in reality, times to first detection are truncated due to a finite survey length of $C$, meaning that each individual has a detection probability $\pdet=F_T(C)$.  
The conditional distribution of observed detection times then has pdf $f_{T|det}(t)= f_T(t)/F_T(C)$ for $0<t<C$ and cdf $F_{T|det}(t) = \int_0^t f_{T|det}(x) dx$. 
We model the number of individuals for which $t_{sb}$ is actually observed as $n_{s}^{(obs)} \ind \mbox{Binomial}\left(N_{s}, \pdet\right)$.
%$T_{sb|det} \ind f_{T|det}(t_{sb})$ where $T_{sb|det}$ is observed only for those $n_s^{(obs)}$ individuals whose observed time to first detection is less than $C$, i.e. $t_{sb}<C$. 
The instantaneous detection rate, or hazard function, is $h(t) = f_T(t) / [1-F_T(t)]$.

A common choice for TTDD is an exponential distribution, i.e. $T_{sb}\ind \mbox{Exp}(\varphi)$, which imposes a constant first detection rate, i.e. $h(t) = \varphi$.
Choosing another TTDD can allow for a systematic non-constant detection regime. 
For example, to model an observer effect where: (i) the observer's arrival suppresses or stimulates detectable cues, but (ii) organisms acclimate and gradually return to constant detection, a gamma TTDD could be appropriate.
In addition to an exponential and gamma TTDD, we also consider Weibull and lognormal as these distributions are often used in survival analysis. 
To facilitate the later inclusion of fixed and random effects, we use the following rate-based parameterizations: $T\sim \Exp(\varphi), E[T]=1/\varphi$; $T\sim \Ga(\alpha,\varphi), E[T] = \alpha/\varphi$; $T\sim \We(\alpha,\varphi), E[T]=\mathrm{\Gamma}(1+1/\alpha)/\varphi$; and $T\sim \LN(\varphi,\alpha), E[T] = e^{\alpha/2}/\varphi$.  
This parameterization of the lognormal relates to the standard ($\mu, \sigma^2$) parameterization by $\varphi = \exp(-\mu)$ and $\alpha = \sigma^2$.  %$\mu = -\log(\varphi)$
The exponential distribution is a special case of both the gamma and Weibull distributions when $\alpha=1$. 

Using maximum likelihood, we can estimate the parameters in a TTDD from exact times to first detection and thus estimate $\pdet$ and its uncertainty.
With an estimate and uncertainty for $\pdet$, we are in a scenario of a binomial model with unknown $N_s$ and ``known'' $p$ and thus can estimate $N_s$.
In these models, point estimates of $N_s$ are unstable unless $\pdet>0.4$ \citep{Olkin1981}.
One approach to regularizing these estimates is to construct a hierarchical model for the site-specific abundance, e.g. $N_s\ind \Po(\lambda)$ \citep{Raftery1988, Royle2004NMixture}. 
With this assumption, we can decompose $N_{s}$ into observed and unobserved portions: $n^{(obs)} \sim Po(\lambda \pdet)$ and, independently, $n_{s}^{(unobs)} \sim \Po\left(\lambda[1-\pdet]\right)$.
Although alternative distributions could be considered, e.g. negative binomial, our experience with Ovenbird point counts suggests that, after accounting for appropriate explanatory variables, the resulting abundances are likely underdispersed rather than overdispersed, and thus we will use the Poisson assumption here. 


\subsection{Interval-censored times to detection}

Due to the harried process of avian point counts, times to first detection are typically not recorded exactly, but are instead censored into $I$ intervals. 
Let $C_i$ for $i=1,\dots,I$ indicate the right endpoint of the $i$th interval then $C_I$ is the total survey duration and, letting $C_0=0$, the $i$th interval is $(C_{i-1},C_{i}]$. 
Finally, let $n_{si}$ be the number of animals counted during interval $i$ on survey $s$, $n_{s}^{(obs)} = \sum_{i=1}^I n_{si}$, and $\vn_{s}=(n_{s1},\dots,n_{sI})$.
Assuming independence amongst individuals and sites, we have $\vn_{s} \ind \Mult \left(n_{s}^{(obs)}, \vp_{s}\right)$, where $\vp_{s}=(p_{s1},\dots,p_{sI})$ and is calculated from the TTDD: $p_{si} = F_{T|det}(C_i) - F_{T|det}(C_{i-1})$.  

It is common in avian point counts to observe increased detections in the first interval relative to an exponential distribution.
To accommodate this empirical observation, many models of interval-censored removal times define a TTDD with a mixture component to increase the probability of observing individuals in the first interval \citep{Farnsworth2002, Royle2004Generalized, Farnsworth2005, Etterson2009, Reidy2011}.
We specify a mixture TTDD with mixing parameter $\gamma\in[0,1]$, a point-mass during the first observation interval, and a continuous-time detection distribution $\Fm(t)$.  
The mixture TTDD cdf is defined: $F_T(t) = (1-\gamma) + \gamma \Fm(t)$ for $t>0$.
If $\gamma=1$, the non-mixture model is recovered.





\subsection{Incorporating explanatory variables}\label{sec:covariates}

As discussed in Section \ref{sec:data}, explanatory variables are available for sites and for surveys. 
Generally, we suspect that site variables, e.g. habitat, will affect abundance and survey variables, e.g. time of day, will affect detection probability. 
Thus, we allow for incorporating explanatory variables on both the abundance and detection.

To incorporate explanatory variables on abundance, we model the expected survey abundance $\lambda_{s}$ via a mixed effects linear regression with a log link, i.e. $\log (\lambda_{s}) = \vX_{s}^A\vbeta^A + \vZ_{s}^A\vxi^A$ where $\vX_{s}^A$ are explanatory variables, $\vbeta^A$ is a vector of fixed effects, $\vZ_{s}^A$ specifies random effect levels, and $\xi_j^A \ind N(0,\sigma_{A[j]}^2)$ are random effects where $A[j]$ assigns the appropriate variance for the $j$th abundance random effect.  

To incorporate explanatory variables on detection probability, we let the TTDD depend on the explanatory variables through the now site-specific parameter $\varphi_s$. 
Specifically, we model $\log(\varphi_{s}) = \vX_{s}^D\vbeta^D + \vZ_{s}^D\vxi^D$, where $\vX_{s}^D$ are explanatory variables, $\vbeta^D$ is a vector of fixed effects, $\vZ_{s}^D$ specifies random effect levels, $\xi_j^D \ind N(0,\tau_{D[j]}^2)$ are random effects where $D[j]$ assigns the appropriate variance for the $j$th detection random effect.  
For simplicity, we model the shape parameter $\alpha$ as constant across sites.



\subsection{Estimation}

For ease of reference, the final full model is provided in equation \eqref{eq:model} where the conditioning of the TTDD cdf on $\alpha$ and $\varphi_s$ is made explicit.
\begin{align}
n_s^{(obs)} &\ind \Po(\lambda_s \pdet_s) \nonumber\\
\vn_s &\ind \Mult(n_s^{(obs)}, \vp_s); \qquad \vp_s = (p_{s1},\dots,p_{sI}) \nonumber\\
p_{s1} &= \left[(1-\gamma) + \gamma\Fm(C_1|\alpha,\varphi_s)\right]/F_T(C_I) \nonumber\\
p_{si} &= \left[\Fm(C_i|\alpha,\varphi_s) - \Fm(C_{i-1}|\alpha,\varphi_s)\right]/F_T(C_I) \nonumber\\
\log(\lambda_s) &= \vX_{s}^A\vbeta^A + \vZ_{s}^A\vxi^A; \qquad \xi_j^A \ind N(0,\sigma_{A[j]}^2) \nonumber\\
\label{eq:model}\log(\varphi_{s}) &= \vX_{s}^D\vbeta^D + \vZ_{s}^D\vxi^D; \qquad \xi_j^D \ind N(0,\tau_{D[j]}^2)
\end{align}
We adopt a Bayesian approach and therefore require a prior over the model parameters.
To ease construction of a default prior for this model, we standardize all explanatory variables and then construct priors to be diffuse within a reasonable range of values.
Normal prior mean and standard deviation (sd) for the abundance intercept was set at a median abundance of 3 birds per site and a 95\% probability of 0-14 birds present (counted and uncounted).  
Normal prior mean and sd for the detection intercept were chosen so that, based on an intercept-only non-mixture model with $\alpha=1$: (i) median prior detection probability was $p_{s}^{(det)} = 0.50$, and (ii) 95\% of the prior detection probability was within $p_{s}^{(det)} \in (0.01, 1.0)$.  
Normal priors for fixed effect parameters were centered at zero with standard deviations matching the appropriate intercept term.  
All standard deviations and $\alpha$ were given half-Cauchy priors with location 0 and scale 1 for the untruncated Cauchy, and the mixture parameter $\gamma$ was assigned a Unif(0,1) prior in mixture models.
All scalar parameters were assumed independent \emph{a priori}. 



We fit the models by MCMC sampling using the Bayesian statistical software Stan, implemented via the R package \texttt{rstan} version 2.8.0 \citep{Rstan2015}.  
We discarded half of the iterations as warmup and then thinned by 10.  
We monitored convergence of the MCMC chains using Geweke z-score diagnostics \citep{Geweke1991} and reran models if lack of convergence was indicated by a non-normal distribution of the z-scores. 
We reran models if the effective sample size for any parameter was below 1000.  
The number of iterations used depended on the model and is detailed later. 
For most models, we accepted Stan defaults for initial values; however, gamma and Weibull models sometimes failed to run unless care was taken in the specification of initial values.




\jarad{Adam: we should include the Stan models in the supplement.}

%Thus, the link functions in our detection model are:
%\begin{alignat}{3}
%&\text{Exponential:\;} &&\log(\varphi_{s}) &&= \text{Intercept}^D + \vX_{s}^D\vbeta^D + \vZ_{s}^D\vxi^D = -\log(E[T_{sq}])\\
%&\text{Gamma:} &&\log(\varphi_{s}) &&= \text{Intercept}^D + \vX_{s}^D\vbeta^D + \vZ_{s}^D\vxi^D = -\log(E[T_{sq}]) + \log(\alpha)\\
%&\text{Weibull:}  &&\log(\varphi_{s}) &&= \text{Intercept}^D + \vX_{s}^D\vbeta^D + \vZ_{s}^D\vxi^D = -\log(E[T_{sq}]) + \log\left(\Gamma\left(1 + 1/k\right)\right)\\
%&\text{Lognormal:} &&-\mu_{s} &&= \text{Intercept}^D + \vX_{s}^D\vbeta^D + \vZ_{s}^D\vxi^D = -\log(E[T_{sq}]) + \sigma_{det}^2/2
%\end{alignat}

%\subsection{Abundance Model}


% We distinguish between non-constant detection and detection heterogeneity, which can be modeled in similar ways but refer to different mechanisms.  
% We reserve `detection heterogeneity' for differences across population subgroups.  
% It is often modeled using a mixture of detection distributions.  
% \citet{Farnsworth2002} introduce a finite-mixture for detection probabilities distinguishing between easy-to-detect individuals, which are always detected in first observation interval, and hard-to-detect individuals, which are detected at a constant rate.  
%  Heterogeneity at the individual level can be modeled as random effects in the calculation of detection probabilities \citep{DorazioRoyle2003, Mantyniemi2005}.  
% We use `non-constant detection' when detection rates for an organism change with time.  
% Models constructed to address detection heterogeneity may provide reasonable estimates for non-constant detection scenarios and vice versa \citep{Mantyniemi2005}.

%\adam{Individual effects on detection might suffer from identifiability issues. This was the subject of a dust-up between Pledger and Dorazio/Royle from 2003-2005.}


We distinguish two categories of TTDDs: peaked and nonpeaked.  
A peaked TTDD has a mode greater than zero (or $C_1$ for lognormal) while a non-peaked TTDD has a mode of zero (or less than $C_1$).




\ifdic\
\subsection{Model diagnostics}

To assess goodness of fit, we relied on posterior predictive checks \citep{Gelman1996}.  
%Posterior predictive checks assess whether a fitted model can effectively replicate datasets that look like the original data in key aspects, which are measured by check statistics.  
In particular, we are concerned with the goodness of fit for the TTDD distribution, so we defined a check statistic for each observation interval:
\[D(n_i) = \sum\limits_{s} n_{si} \big/ \sum\limits_{si} n_{si}\]
which is the proportion of the total count that occurs during interval $i$ marginally across all surveys.  
The posterior predictive p-value associated with each $D(n_i)$ is the proportion of replicate check statistics that have a larger value than the check statistic for the actual data.  
Posterior predictive p-values near to 0 or 1 are potentially indicators of a misspecified model.
We simulated replicate datasets and calculated the check statistic at each iteration of our MCMC sampling.  

We focus inference on the overall detection probability across all surveys: $\pdet = \sum\limits_{s}n_{s}^{(obs)}\big/$ $\sum\limits_{s}N_{s}$.  
Within simulation studies, we compared model estimates to true $\pdet$ values by percent coverage and by posterior p-values --- the proportion of MCMC samples having a smaller value than the true value.  

We compared different models by means of their goodness of fit and deviance information criterion (DIC) \citep{Spiegelhalter2002}.  
For the full model with random effects (Ovenbirds and Sim3), which is a missing data model, we used formulation DIC$_7$ from \citet{Celeux2006} for its ease of computation, but we acknowledge that it suffers some difficulties \citep{Celeux2006, Li2014}.
\fi




\section{Simulation studies} \label{sec:sim}

We conducted three simulation studies to explore the behavior of models with non-constant TTDDs.
The first study compares mixture vs non-mixture models.
The second study compares the time to detection distribution families.
In the first two studies, we utilized intercept only models to focus attention on the time to detection distribution. 
For the third study, we included fixed and random effects for both abundance and detection and again compared the distribution families. 
In all simulation studies, we focus on accuracy in estimation of $\pdet$ which then translates into estimation of abundance. 









\subsection{Mixture versus non-mixture TTDDs}\label{sec:mixture}

In Sim1, we simulated 16 intercept-only datasets from each of 14 TTDDs. 
We chose true parameter values (see Supplement) so that (i) the overall expected detection probability was 0.80, (ii) for mixture datasets, $\gamma = 0.65$ (meaning 35\% of individuals were immediately detected), (iii) in nonpeaked models, 70\% of \textit{detected} individuals were observed during the first two minutes, and (iv) in peaked models, the detection mode for `hard to detect' individuals occured at 5 minutes.
These values are largely consistent with estimates of Ovenbird populations from Section \ref{sec:ovenbirds}.

Each dataset was fit with two models: mixture and non-mixture version of the distribution family, e.g. exponential, used to simulate the data.
We ran 60,000 iterations which showed no evidence of lack of convergence according to the Geweke diagnostic and reached over 1,000 effective samples for all parameters.

Table \ref{tbl:sim1} provides a summary of the mixture and non-mixture models' ability to capture the detection probability. 
When a non-mixture model is used to simulate the data (top half of the table), there are no clearly discernable differences between the ability of a non-mixture or mixture model to capture.  \adam{For nonpeaked non-mixture data, the mixture model is less biased for lognormal and Weibull scenarios, and CIs are 15-40\% narrower.}

\ifdic
\begin{table}[ht]\centering\small
\begin{tabular}{l|l|l|l|cccc|cccc||r}
 \multicolumn{4}{c|}{ } & \multicolumn{4}{c|}{\underline{Non-mixture model}} & \multicolumn{4}{c||}{\underline{Mixture model}} & \\
 \multicolumn{4}{c|}{ } & Med. $p$ & Q($p$) & 50\% & 90\% & Med. $p$ & Q($p$) & 50\% & 90\% & $\Delta$ DIC \\ 
  \hline
  \hline
 \parbox[t]{2mm}{\multirow{14}{*}{\rotatebox[origin=c]{90}{TTDD used to simulate data}}} & \parbox[t]{2mm}{\multirow{7}{*}{\rotatebox[origin=c]{90}{Non-mixture}}} & \parbox[t]{2mm}{\multirow{3}{*}{\rotatebox[origin=c]{90}{Nonpk.}}} & Gamma & 0.76 & 0.41 & 0.75 & 0.88 & 0.84 & 0.66 & 0.56 & 0.94 & 0.24 \\ 
 & & &   Lognormal & 0.66 & 0.17 & 0.31 & 0.62 & 0.78 & 0.48 & 0.62 & 1.00 & -0.67 \\ 
 & & &   Weibull & 0.69 & 0.25 & 0.38 & 0.75 & 0.79 & 0.51 & 0.75 & 1.00 & 0.37 \\ 
  \cline{3-13}
 & & &   Exponential & 0.80 & 0.54 & 0.44 & 0.94 & 0.79 & 0.38 & 0.31 & 0.88 & 1.48 \\ 
  \cline{3-13}
 & & \parbox[t]{2mm}{\multirow{3}{*}{\rotatebox[origin=c]{90}{Peaked}}} &  Gamma & 0.80 & 0.55 & 0.50 & 1.00 & 0.82 & 0.70 & 0.38 & 0.88 & 0.96 \\ 
 & & &   Lognormal & 0.78 & 0.31 & 0.50 & 0.94 & 0.80 & 0.48 & 0.62 & 1.00 & 0.97 \\ 
 & & &   Weibull & 0.79 & 0.49 & 0.56 & 0.94 & 0.82 & 0.66 & 0.44 & 0.88 & 1.28 \\ 
  \cline{2-13}
& \parbox[t]{2mm}{\multirow{7}{*}{\rotatebox[origin=c]{90}{Mixture}}} & \parbox[t]{2mm}{\multirow{3}{*}{\rotatebox[origin=c]{90}{Nonpk.}}} & Gamma & 0.67 & 0.17 & 0.12 & 0.81 & 0.76 & 0.41 & 0.69 & 1.00 & 0.51 \\ 
 & & &   Lognormal & 0.56 & 0.04 & 0.00 & 0.31 & 0.72 & 0.30 & 0.44 & 0.88 & 0.33 \\ 
 & & &   Weibull & 0.51 & 0.02 & 0.00 & 0.06 & 0.71 & 0.32 & 0.44 & 1.00 & 2.40 \\ 
  \cline{3-13}
 & & &   Exponential & 0.96 & 1.00 & 0.00 & 0.00 & 0.77 & 0.37 & 0.38 & 0.94 & 122.81 \\ 
  \cline{3-13}
 & & \parbox[t]{2mm}{\multirow{3}{*}{\rotatebox[origin=c]{90}{Peaked}}} & Gamma & 0.28 & 0.00 & 0.00 & 0.00 & 0.74 & 0.33 & 0.31 & 0.88 & 31.84 \\ 
 & & &   Lognormal & 0.22 & 0.00 & 0.00 & 0.00 & 0.76 & 0.36 & 0.38 & 0.94 & 63.65 \\ 
 & & &   Weibull & 0.22 & 0.00 & 0.00 & 0.00 & 0.70 & 0.29 & 0.56 & 0.94 & 28.30 \\ 
   \hline
\end{tabular}
\caption{Summary of mixture vs. non-mixture model fits (Sim1).  
In all cases, the inference model family matches the dataset family.  
Med $p$: average across simulations of the posterior median of $\pdet$ (true value = 0.80).  
Q($p$): average proportion of the posterior distribution of $\pdet$ that is larger than the true value.  
50\% and 90\% coverage is expressed as the proportion of 16 simulations for which the true value of $\pdet$ lies within the appropriate credible interval.  
$\Delta$ DIC: mean difference in DIC between the incorrect-mixture model minus the true model.
}
\label{tbl:sim1}
\end{table}

\else

\begin{table}[ht]\centering\small
\begin{tabular}{l|l|l|l|cccc|cccc}
 \multicolumn{4}{c|}{ } & \multicolumn{4}{c|}{\underline{Non-mixture model}} & \multicolumn{4}{c}{\underline{Mixture model}} \\
 \multicolumn{4}{c|}{ } & Med. $p$ & Q($p$) & 50\% & 90\% & Med. $p$ & Q($p$) & 50\% & 90\% \\ 
  \hline
  \hline
 \parbox[t]{2mm}{\multirow{14}{*}{\rotatebox[origin=c]{90}{TTDD used to simulate data}}} & \parbox[t]{2mm}{\multirow{7}{*}{\rotatebox[origin=c]{90}{Non-mixture}}} & \parbox[t]{2mm}{\multirow{3}{*}{\rotatebox[origin=c]{90}{Nonpk.}}} & Gamma & 0.76 & 0.41 & 0.75 & 0.88 & 0.84 & 0.66 & 0.56 & 0.94 \\ 
 & & &   Lognormal & 0.66 & 0.17 & 0.31 & 0.62 & 0.78 & 0.48 & 0.62 & 1.00  \\ 
 & & &   Weibull & 0.69 & 0.25 & 0.38 & 0.75 & 0.79 & 0.51 & 0.75 & 1.00 \\ 
  \cline{3-12}
 & & &   Exponential & 0.80 & 0.54 & 0.44 & 0.94 & 0.79 & 0.38 & 0.31 & 0.88  \\ 
  \cline{3-12}
 & & \parbox[t]{2mm}{\multirow{3}{*}{\rotatebox[origin=c]{90}{Peaked}}} &  Gamma & 0.80 & 0.55 & 0.50 & 1.00 & 0.82 & 0.70 & 0.38 & 0.88 \\ 
 & & &   Lognormal & 0.78 & 0.31 & 0.50 & 0.94 & 0.80 & 0.48 & 0.62 & 1.00 \\ 
 & & &   Weibull & 0.79 & 0.49 & 0.56 & 0.94 & 0.82 & 0.66 & 0.44 & 0.88 \\ 
  \cline{2-12}
& \parbox[t]{2mm}{\multirow{7}{*}{\rotatebox[origin=c]{90}{Mixture}}} & \parbox[t]{2mm}{\multirow{3}{*}{\rotatebox[origin=c]{90}{Nonpk.}}} & Gamma & 0.67 & 0.17 & 0.12 & 0.81 & 0.76 & 0.41 & 0.69 & 1.00 \\ 
 & & &   Lognormal & 0.56 & 0.04 & 0.00 & 0.31 & 0.72 & 0.30 & 0.44 & 0.88 \\ 
 & & &   Weibull & 0.51 & 0.02 & 0.00 & 0.06 & 0.71 & 0.32 & 0.44 & 1.00 \\ 
  \cline{3-12}
 & & &   Exponential & 0.96 & 1.00 & 0.00 & 0.00 & 0.77 & 0.37 & 0.38 & 0.94 \\ 
  \cline{3-12}
 & & \parbox[t]{2mm}{\multirow{3}{*}{\rotatebox[origin=c]{90}{Peaked}}} & Gamma & 0.28 & 0.00 & 0.00 & 0.00 & 0.74 & 0.33 & 0.31 & 0.88 \\ 
 & & &   Lognormal & 0.22 & 0.00 & 0.00 & 0.00 & 0.76 & 0.36 & 0.38 & 0.94 \\ 
 & & &   Weibull & 0.22 & 0.00 & 0.00 & 0.00 & 0.70 & 0.29 & 0.56 & 0.94 \\ 
   \hline
\end{tabular}
\caption{Summary of mixture vs. non-mixture model fits (Sim1).  
In all cases, the inference model family matches the dataset family.  
Med $p$: average across simulations of the posterior median of $\pdet$ (true value = 0.80).  
Q($p$): average proportion of the posterior distribution of $\pdet$ that is larger than the true value.  
50\% and 90\% coverage is expressed as the proportion of 16 simulations for which the true value of $\pdet$ lies within the appropriate credible interval.  
\jarad{We should consider splitting this table in two. The main point is what happens when the data arise from a mixture model but a non-mixture distribution is used. So the bottom half could be included in the main body while the top half could be put in the supplement.}
\adam{The obvious concern for non-mixture data is that the mixture model might be more uncertain than a non-mixture model for non-mixture data.}
}
\label{tbl:sim1}
\end{table}
\fi

When a mixture model is used to simulate the data (bottom half of the table), there is clearly a benefit to using a mixture inference model.  
For the non-mixture models, the credible interval coverage is near zero for most models with the exponential model overestimating $\pdet$ and the other models underestimating. 

These results support the use of a mixture model unless you are absolutely sure your data arose from a non-mixture distribution.


% Median posterior abundance from mixture models was unbiased for non-mixture data and positively biased by only 4-15\% for mixture data.  
% Median posterior abundance from non-mixture models applied to non-mixture data were unbiased (exponential and peaked) or biased by 5-24\% (nonpeaked); however, for mixture data, the non-mixture models were biased by 20-60\% (nonpeaked), -17\% (exponential), and 183-265\% (peaked).  
% For the time-varying distributions, non-mixture models not only had greater bias, but 95\% credible intervals were generally 10-40\% wider (470\% wider for peaked mixture data).
% For exponential datasets, the non-mixture model credible intervals were 28\% narrower for non-mixture data and an unrealistic 95\% narrower for mixture data.

% Possible other comments
%(1) For nonpeaked data, the true inference model is biased low\\
%(2) Mixture inference models generate higher estimates than non-mixture models\\
%(3) True models are generally good\\
%(4) Estimates of non-mixture peaked data are much more precise\\
%(5) Note that plots of TTDD at the median are indistinguishable from the truth except in the blatant fail scenario







\subsection{Constant vs. non-constant detection mixture TTDDs}\label{sec:family}

The previous section adressed model mis-specification in terms of the mixture component. 
Now we turn to model misspecification of the distribution family. 
For Sim2, we simulated 16 intercept-only datasets from the 7 different mixture TTDD models using the same settings as in the previous section, and we fit them with mixture models from each of exponential, gamma, lognormal, and Weibull families.
%Again, we ran 60,000 iterations which showed no evidence of lack of convergence according to the Geweke diagnostic and reached over 1,000 effect samples for all parameters.

Table \ref{tbl:sim2} provides a summary of the ability of each model to capture the detection probability. 
Each quadrant assesses a model used for inference compared to the 7 different models used for simulation. 
The poorest estimation of $\pdet$ occurs for the exponential inferential model when the model used for simulation has a peak, because the two parameters (rate and mixing parameter) do not provide enough flexibility for the mixture exponential distribution to adequately fit a TTDD with both an initial increase and a delayed mode. 
In this situation, the exponential model underestimates the actual detection probability.
In contrast, if the simulated data are non-peaked, the exponential model typically overestimates the actual detection probability. 
\adam{For exponential data, the exponential model performs better.  I feel this should be noted.}

\ifdic
\begin{table}[ht]
\footnotesize\centering
\begin{tabular}{l|l|l|ccccc|ccccc}
 \multicolumn{3}{c}{ } & \multicolumn{5}{c}{\underline{Exponential mixture model}} & \multicolumn{5}{c}{\underline{Gamma mixture model}} \\
 \multicolumn{3}{c}{ } & Med. $p$ & Q($p$) & 50\% & 90\% & $\Delta$ DIC & Med. $p$ & Q($p$) & 50\% & 90\% & $\Delta$ DIC \\ 
  \hline
\parbox[t]{2mm}{\multirow{7}{*}{\rotatebox[origin=c]{90}{Data Mixture}}} & \parbox[t]{2mm}{\multirow{3}{*}{\rotatebox[origin=c]{90}{Nonpk.}}} & Gamma & 0.88 & 0.91 & 0.06 & 0.69 & 1.54 & 0.76 & 0.41 & 0.69 & 1.00 & --- \\ 
& &   Lognormal & 0.92 & 0.99 & 0.00 & 0.12 & 2.08 & 0.85 & 0.72 & 0.44 & 0.69 & 0.93 \\ 
& &   Weibull & 0.87 & 0.83 & 0.31 & 0.38 & 0.67 & 0.79 & 0.49 & 0.69 & 1.00 & -0.08 \\ 
\cline{2-13}
& &   Exponential & 0.77 & 0.37 & 0.38 & 0.94 & --- & 0.68 & 0.24 & 0.25 & 0.81 & -0.97 \\ 
\cline{2-13}
& \parbox[t]{2mm}{\multirow{3}{*}{\rotatebox[origin=c]{90}{Peaked}}} & Gamma & 0.36 & 0.00 & 0.00 & 0.00 & 6.63 & 0.74 & 0.33 & 0.31 & 0.88 & --- \\ 
& &   Lognormal & 0.34 & 0.00 & 0.00 & 0.00 & 18.54 & 0.84 & 0.73 & 0.44 & 0.81 & 0.71 \\ 
& &   Weibull & 0.39 & 0.00 & 0.00 & 0.00 & 1.91 & 0.66 & 0.15 & 0.19 & 0.75 & -0.11 \\
   \hline
\end{tabular}
\vspace{0.5cm}\\
\begin{tabular}{l|l|l|ccccc|ccccc}
 \multicolumn{3}{c}{ } & \multicolumn{5}{c}{\underline{Lognormal mixture model}} & \multicolumn{5}{c}{\underline{Weibull mixture model}} \\
 \multicolumn{3}{c}{ } & Med. $p$ & Q($p$) & 50\% & 90\% & $\Delta$ DIC & Med. $p$ & Q($p$) & 50\% & 90\% & $\Delta$ DIC \\ 
  \hline
\parbox[t]{2mm}{\multirow{7}{*}{\rotatebox[origin=c]{90}{Data Mixture}}} & \parbox[t]{2mm}{\multirow{3}{*}{\rotatebox[origin=c]{90}{Nonpk.}}} & Gamma & 0.65 & 0.16 & 0.19 & 0.81 & 0.51 & 0.67 & 0.24 & 0.31 & 1.00 & 0.03 \\ 
& &   Lognormal & 0.72 & 0.30 & 0.44 & 0.88 & --- & 0.77 & 0.48 & 0.44 & 1.00 & 0.36 \\ 
& &   Weibull & 0.68 & 0.23 & 0.31 & 0.88 & 0.38 & 0.71 & 0.32 & 0.44 & 1.00 & --- \\ 
\cline{2-13}
& &   Exponential & 0.61 & 0.12 & 0.12 & 0.44 & -0.19 & 0.63 & 0.20 & 0.25 & 0.75 & -0.62 \\ 
\cline{2-13}
& \parbox[t]{2mm}{\multirow{3}{*}{\rotatebox[origin=c]{90}{Peaked}}} & Gamma & 0.65 & 0.13 & 0.06 & 0.56 & -0.72 & 0.79 & 0.52 & 0.69 & 0.88 & 0.27 \\ 
& &   Lognormal & 0.76 & 0.36 & 0.38 & 0.94 & --- & 0.89 & 0.88 & 0.12 & 0.69 & 1.52 \\ 
& &   Weibull & 0.58 & 0.03 & 0.00 & 0.12 & -0.50 & 0.70 & 0.29 & 0.56 & 0.94 & --- \\ 
   \hline
\end{tabular}
\caption{Summary of mixture inference models of all families fit to mixture datasets.  
Med $p$: average across simulations of the posterior median of $\pdet$ (true value = 0.80).  
Q($p$): average proportion of the posterior distribution of $\pdet$ that is larger than the true value.  
50\% and 90\% coverage is expressed as the proportion of simulations for which the true value of $\pdet$ lies within the appropriate credible interval.  
$\Delta$ DIC: mean difference in DIC between the incorrect-family model minus the true model.}
\label{tbl:sim2}
\end{table}

\else

\begin{table}[ht]
\footnotesize\centering
\begin{tabular}{l|l|l|cccc|cccc}
 \multicolumn{3}{c}{ } & \multicolumn{4}{c}{\underline{Exponential mixture model}} & \multicolumn{4}{c}{\underline{Gamma mixture model}} \\
 \multicolumn{3}{c}{ } & Med. $p$ & Q($p$) & 50\% & 90\%  & Med. $p$ & Q($p$) & 50\% & 90\% \\ 
  \hline
\parbox[t]{2mm}{\multirow{7}{*}{\rotatebox[origin=c]{90}{Data Mixture}}} & \parbox[t]{2mm}{\multirow{3}{*}{\rotatebox[origin=c]{90}{Nonpk.}}} & Gamma & 0.88 & 0.91 & 0.06 & 0.69 & 0.76 & 0.41 & 0.69 & 1.00 \\ 
& &   Lognormal & 0.92 & 0.99 & 0.00 & 0.12 & 0.85 & 0.72 & 0.44 & 0.69 \\ 
& &   Weibull & 0.87 & 0.83 & 0.31 & 0.38 & 0.79 & 0.49 & 0.69 & 1.00 \\ 
\cline{2-11}
& &   Exponential & 0.77 & 0.37 & 0.38 & 0.94 & 0.68 & 0.24 & 0.25 & 0.81 \\ 
\cline{2-11}
& \parbox[t]{2mm}{\multirow{3}{*}{\rotatebox[origin=c]{90}{Peaked}}} & Gamma & 0.36 & 0.00 & 0.00 & 0.00 & 0.74 & 0.33 & 0.31 & 0.88 \\ 
& &   Lognormal & 0.34 & 0.00 & 0.00 & 0.00 & 0.84 & 0.73 & 0.44 & 0.81 \\ 
& &   Weibull & 0.39 & 0.00 & 0.00 & 0.00 & 0.66 & 0.15 & 0.19 & 0.75 \\
   \hline
\end{tabular}
\vspace{0.5cm}\\
\begin{tabular}{l|l|l|cccc|cccc}
 \multicolumn{3}{c}{ } & \multicolumn{4}{c}{\underline{Lognormal mixture model}} & \multicolumn{4}{c}{\underline{Weibull mixture model}} \\
 \multicolumn{3}{c}{ } & Med. $p$ & Q($p$) & 50\% & 90\% & Med. $p$ & Q($p$) & 50\% & 90\% \\ 
  \hline
\parbox[t]{2mm}{\multirow{7}{*}{\rotatebox[origin=c]{90}{Data Mixture}}} & \parbox[t]{2mm}{\multirow{3}{*}{\rotatebox[origin=c]{90}{Nonpk.}}} & Gamma & 0.65 & 0.16 & 0.19 & 0.81 & 0.67 & 0.24 & 0.31 & 1.00 \\ 
& &   Lognormal & 0.72 & 0.30 & 0.44 & 0.88 & 0.77 & 0.48 & 0.44 & 1.00 \\ 
& &   Weibull & 0.68 & 0.23 & 0.31 & 0.88 & 0.71 & 0.32 & 0.44 & 1.00 \\ 
\cline{2-11}
& &   Exponential & 0.61 & 0.12 & 0.12 & 0.44 & 0.63 & 0.20 & 0.25 & 0.75 \\ 
\cline{2-11}
& \parbox[t]{2mm}{\multirow{3}{*}{\rotatebox[origin=c]{90}{Peaked}}} & Gamma & 0.65 & 0.13 & 0.06 & 0.56 & 0.79 & 0.52 & 0.69 & 0.88 \\ 
& &   Lognormal & 0.76 & 0.36 & 0.38 & 0.94 & 0.89 & 0.88 & 0.12 & 0.69 \\ 
& &   Weibull & 0.58 & 0.03 & 0.00 & 0.12 & 0.70 & 0.29 & 0.56 & 0.94 \\ 
   \hline
\end{tabular}
\caption{Summary of mixture inference models of all families fit to mixture datasets.  
Med $p$: average across simulations of the posterior median of $\pdet$ (true value = 0.80).  
Q($p$): average proportion of the posterior distribution of $\pdet$ that is larger than the true value.  
50\% and 90\% coverage is expressed as the proportion of simulations for which the true value of $\pdet$ lies within the appropriate credible interval.  
$\Delta$ DIC: mean difference in DIC between the incorrect-family model minus the true model.}
\label{tbl:sim2}
\end{table}
\fi

When comparing the different three-parameter TTDDs, model misspecification is not as serious an issue because these models can adequately account for the interval-censored times to detection. 
Nonetheless, it appears the gamma mixture model is better able to account for data from lognormal or Weibull mixture models. 
In the gamma quadrant of Table \ref{tbl:sim2}, the average posterior median is near the truth of 0.8, the average proportion of the posterior distribution of $\pdet$ is near 0.5, and the credible intervals have coverage near their credibility. 
In contrast, when the lognormal model is used for inference, it performed worse when the data arose from a gamma or Weibull and, when the Weibull model was used for inference, it performed worse when the data arose from a lognormal.
Other than sometimes providing evidence against the exponential mixture model, DIC was not helpful in distinguishing amongst these models. 
\adam{I'd be okay with jettisoning this paragraph beyond the first sentence.  Conversely, rather than praising gamma, we could discourage lognormal.}

These results support the use of a non-constant rate mixture TTDD model unless you are absolutely sure your data arose from an exponential mixture TTDD.


% Mixture exponential models returned more accurate estimates of $\pdet$ than other models when data came from an exponential mixture, and 95\% credible intervals for $\pdet$ were from one-third to one-half as wide.  
% However, mixture exponential models yielded notably biased estimates with poor coverage for all non-exponential data, overestimating detection for nonpeaked datasets and underestimating it for peaked datasets.  
% Differences among non-constant TTDDs were less pronounced.  
% Gamma models were more accurate than others for nonpeaked mixture datasets.  
% Gamma and Weibull models seemed equally accurate for peaked mixture datasets.  
% Also, 95\% credible intervals for gamma models were 5-15\% narrower than for Weibull models and 15-25\% narrower than for lognormal models.
% 
% Median posterior abundance for exponential models were positiviely biased only 4\% for exponential data, while time-varying TTDD estimates were biased from 19-34\% with average 95\% credible intervals being 3.5 to 5.5 times wider.
% For nonpeaked mixture data, posterior median abundance was negatively biased 8-13\% for exponential models, was unbiased for gamma models and had bias of 5-20\% for Weibull and 13-23\% for lognormal.
% So, in terms of raw abundance, the magnitude of bias for nonpeaked data from exponential models was commensurate with magnitudes from time-varying TTDD models; however, credible intervals from exponential models were only 10-30\% as wide as from other models, thus explaining the poor coverage in Table \ref{tbl:sim2}.
% For peaked mixture data, exponential model median posterior abundance was biased 126\%, while time-varying TTDD estimates were only 3-23\% biased; exponential model credible intervals were from 2-12 times wider than from time-varying TTDD models.
% 
% In most cases, neither DIC nor posterior predictive checks evidenced preference for any model, even when bias and coverage suggested one was inferior.  For  exponential mixture fits of gamma and lognormal peaked datasets, DIC did show a clear difference, and posterior predictive checks revealed lack-of-fit with p-values of 1.00 in the second observation interval.





\subsection{Models including covariates and random effects}\label{sec:simfull}

The previous sections studied effects of time to detection assumptions in the context of no explanatory variables.
In Sim3, we incorporate fixed and random effects for abundance and detection. 
We simulated data from each of the 7 mixture TTDDs and fit models from exponential, gamma, lognormal, and Weibull mixture models.  
We simulated data using the median posterior parameter estimates obtained in the analysis of Ovenbird data in Section \ref{sec:ovenbirds}.
Because the fitted Ovenbird models did not yield peaked distributions, we simulated peaked datasets by: (i) using the same intercepts, shape parameters, and mixing parameters as for peaked data in Sim1 and Sim2, (ii) using median covariate and random effects from the Ovenbird estimates, and (iii) scaling the detection intercept and random effect to achieve true detection probabilities $\approx 0.8$ with a detection mode at 5 minutes.
(See Table \ref{tbl:sim3} in supplement for actual parameter values).
Due to the computation time involved in estimating models with these fixed and random effects, we simulated each TTDD only once.
To obtain reasonable convergence diagnostics and effective sample sizes, these analyses ranged from 250,000-375,000 iterations.

Patterns in posterior estimates of $\pdet_s$ with respect to mixture and family TTDD forms were the same in Sim3 as in Sim1 and Sim2 -- the inclusion of explanatory variables did not make models more robust to violations of mixture- and constant-detection assumptions (Figures \ref{pdet_cater_correct} and \ref{pdet_cater_family} in the Supplement).  
Posteriors for abundance fixed and random effects were the same regardless of what TTDD was assumed (see Supplemental figure).  
Posteriors for the mixing parameter $\gamma$ and detection fixed and random effects were the same across gamma, lognormal, and Weibull mixture models but were narrower and location-shifted for the exponential mixture model.  
It should be remembered that detection covariate effect estimates are conditional on the estimated mixing parameter $\gamma$.  
The above patterns can also be discerned in posterior estimates from the Ovenbird data (Figure \ref{ovenposteriors}).

%DIC can be effective in the same scenarios as Sim1 and Sim2, but it can also be misleading.  
%For peaked mixture data, mixture models were preferred over non-mixture models (DIC differences $>10$), and time-varying mixture TTDDs were preferred over the exponential mixture TTDD for gamma and lognormal peaked data ($>8$).
%For exponential mixture data, DIC preferred the exponential mixture TTDD to time-varying mixture TTDDs.  
%However, DIC also preferred the exponential mixture to time-varying mixtures (by 4-8) for lognormal nonpeaked data.  
%In this last case, the posterior 95\% credible interval for $\pdet$ from the exponential mixture model did not contain the true value of $\pdet$, while the true value was within a central 60\% credible interval for all three other mixture models.  
%In conjunction with Sim2 results, this last finding calls into question the utility of DIC for nonpeaked datasets.
%DIC performance in Sim3 may differ from Sims 1 \& 2 not only because of the addition of fixed and random effects but also because DIC was calculated according to a different method.





\ifdic
\subsection{Model Fits}\label{sec:fits}

In Sim1, DIC and posterior predictive checks could not easily differentiate a mixture from a non-mixture model for inference when a nonpeaked simulation model was used, even though non-mixture models displayed strong bias and poor coverage.
Both tools clearly preferred mixture models when the data arose from a mixture exponential distribution or any of the peaked mixture distributions. 
In these cases, posterior predictive p-values demonstrated that, relative to the observed data, the estimated TTDD placed too many counts during these intervals and too few in the first and later intervals.
\fi





\section{Ovenbird analysis}\label{sec:ovenbirds}

We fit the Ovenbird dataset with exponential, gamma, lognormal, and Weibull mixture models.  
For the abundance half of our model, we used four covariates plus two random effects.  
The covariates were: (a) site age, (b) survey year, (c) an indicator of whether the site stock density was over 70\%, and (d) an indicator of whether the site experienced select-/partial-cut logging during the 1990s.  
We associated random effects with each survey year and each stand.
For the detection half of our model, we used covariates for: (a) Julian date, (b) time of day, (c) temperature, (d) an indicator of whether it is the observer's first year in the database, and (e) an interaction between (a) and (d) to approximate a new observer's learning curve.  
We associated random effects with each observer.  
Preliminary model fits did not support the inclusion of quadratic terms for any detection covariates.  
We centered and standardized all continuous covariates prior to fitting models.
We ran chains 250,000-375,000 iterations; Geweke diagnostics showed no indication of lack of fit, and effecive sample sizes were over 1000 for all parameters.


% Effects: * - means signif
% Observer variation - Farnsworth2002, LinkSauer98, LinkSauer97, Dief07 cites Sauer94 & Dief03, Simons et al03
% First-year observer - LinkSauer98
% Time of day - Farnsworth2002, Soly13*, Amundson
% Season (jdate) - Farnsworth2002, Soly13*, Amundson, Dief07
% Quadratic effects - Soly13
% Tree cover (density) - Soly13
% Year (on detection!) - Reidy11, see also Norvell03
% See Table 1 of Johnson2008, see McShea and Rappole, lit review from Warren2013, Rosenstock02

%\begin{table}[ht]
%\centering
%\begin{tabular}{lccccccccc}
%  \hline
%Minutes & 0-2 & 2-3 & 3-4 & 4-5 & 5-6 & 6-7 & 7-8 & 8-9 & 9-10 \\ 
%Count & 596 & 69 & 62 & 68 & 43 & 33 & 35 & 27 & 14 \\ 
%\hline
%\multicolumn{10}{l}{Posterior predictive check $p$-values by model:}\\
%\hline
%Expo & 0.000 & 1.000 & 1.000 & 0.347 & 0.590 & 0.372 & 0.007 & 0.006 & 0.190 \\ 
%  Gamma & 0.582 & 0.820 & 0.394 & 0.006 & 0.362 & 0.637 & 0.241 & 0.519 & 0.977 \\ 
%  LogNormal & 0.571 & 0.943 & 0.471 & 0.005 & 0.322 & 0.554 & 0.165 & 0.407 & 0.974 \\ 
%  Weibull & 0.582 & 0.862 & 0.401 & 0.005 & 0.359 & 0.616 & 0.221 & 0.509 & 0.982 \\ 
%\hline
%  Exponential Mix & 0.533 & 0.804 & 0.577 & 0.021 & 0.501 & 0.668 & 0.190 & 0.338 & 0.918 \\ 
%  Gamma Mix & 0.593 & 0.720 & 0.431 & 0.015 & 0.439 & 0.670 & 0.246 & 0.476 & 0.947 \\ 
%  LogNormal Mix & 0.576 & 0.806 & 0.508 & 0.016 & 0.421 & 0.627 & 0.197 & 0.413 & 0.951 \\ 
%  Weibull Mix & 0.596 & 0.734 & 0.411 & 0.013 & 0.421 & 0.663 & 0.252 & 0.492 & 0.958 \\ 
%   \hline
%\end{tabular}
%\caption{\label{tbl:ovencounts} Ovenbird counts and posterior predictive p-values -- Pr$\left(D(n_i)^{(rep)} > D(n_i)^{(obs)}\right)$ -- by observation interval for each model across all surveys.  P-values near zero or one may indicate poor model fit.}
%\end{table}

Figure \ref{ovenposteriors} presents caterpillar plots of posterior estimates for model parameters, overall detection probability $\pdet$, and the log10 number of Ovenbirds uncounted.  
Based on shape parameter estimates and patterns in the estimation of $\gamma$, $\beta^D$, $\pdet$, and total abundance across models, Ovenbird data most likely arose from an exponential or nonpeaked mixture TTDD.

As in Sim3, abundance covariate coefficient estimates were virtually the same across all TTDD models.  
95\% credible intervals for two of the abundance parameters do not contain zero, thereby suggesting notable effects.  
Select- and partial-cut logging events of the 1990s depressed local Ovenbird abundance during the study perior to roughly 25-50\% of the abundance for unlogged sites.  
Credible intervals for Site Age indicate that each decade of age increases abundance from 1.5-13\%.
Credible intervals for detection parameters do not indicate significant effects for any of the predictors.

%DIC calculations preferred the exponential mixture TTDD to the lognormal, Weibull, and gamma mixtures (differences of 6.95, 10.7, and 11.5, respectively).  
%These differences closely mirror those from the exponential mixture dataset in Sim3, but they also mirror those from the nonpeaked lognormal mixture dataset, where the exponential mixture 95\% credible interval for abundance did not contain the true value, and its posterior median was negatively biased by 18\% of the actual abundance.

In spite of the similarity of effect parameter estimates, the posterior distributions for detection probability and uncounted abundance differ greatly between the exponential and non-exponential models.  
It is clear that the assumption of constant detection leads to much higher and more precise estimates of detection than would be obtained if we are unwilling to make that assumption.
%The Ovenbird analysis highlights the ambiguity that emerged from the simulation studies: the results were entirely consistent with data generated from a finite-mixture involving a constant-rate $F_{T1}$ TTDD, but they were also consistent with finite-mixtures involving a time-varying TTDD.



\begin{figure}[h!]\centering
\includegraphics[width=0.95\textwidth]{OVEN/oven_sum/OVEN_Posteriors.pdf}
\caption{\label{ovenposteriors} Caterpillar plots of posterior parameter estimates for models fit to Ovenbird data.  
Abundance and detection parameters are identified by (A) and (D), respectively.  
Gamma is the mixing parameter.  
Shape(Alpha) are shape parameters for gamma, Weibull, and lognormal distributions.  
p\_det and log10(Uncounted) are the detection probabilty and the number of birds present but not counted across all surveys.  
Black bars are 95\% credible intervals, orange bars are 50\% credible intervals, and black dots are posterior medians.}
\end{figure}






\section{Discussion} \label{sec:discuss}

Historically, analysis of removal-sampled data assumes constant detection rates throughout the sampling period.  
This assumption provides an intuitive null model and is analytically tractable.  
However, although analysis of removal models has grown more sophisticated in recent years, the constant-detection assumption nonetheless continues unquestioned.  
We have formulated a model for non-constant detection using a time-to-event approach within a hierarchical N-mixture framework.  
Our results demonstrate that, for better or worse, the assumption of constant detection is a very strong assumption.  
When it is correct, it is the most accurate and precise of TTDD models, but when detection rates are not constant, it results in significant bias.  
Models with non-constant TTDDs, on the other hand, can adapt to a wider variety of detection patterns.

We modeled two varieties of heterogeneous detection in conjunction with one other: (i) time-varying detection rates, as modeled by non-exponential TTDDs, and (ii) detection heterogeneity among subgroups of individuals, as modeled by mixture TTDDs.  
The exponential TTDD, even when modeled with a mixture component, results in biased estimates of detection when detection actually varies with time.  
Gamma, Weibull, and lognormal TTDDs, with their extra parameter, are more accurate in handling non-constant detection data, though they are less precise when estimating data with constant detection.

The utility of applying a mixture TTDD for constant-detection models has been identified several times \citep{Pledger2000, Farnsworth2002, Alldredge2007, Reidy2011}.  
The mixture allows an analysis to account for detection heterogeneity across easy- and hard-to-detect subgroups of a population.  
Even so, some recent studies have assumed the absence of a mixture \citep{Solymos2013, Amundson2014}.  
Our study indicates that mixtures are useful even when a non-constant TTDD is used for hard-to-detect individuals.  
Models that include a mixture generally perform well even when no heterogeneity is present, but models that omit a mixture are biased and overly certain when detection heterogeneity exists.  
Indeed, for peaked datasets, mixture models can outperform non-mixture models even in the absence of heterogeneity.
% ``It is tempting to choose classes of models that yield narrower confidence intervals for N; however, Coull's (1997) extensive simulation studies (summarized by Coull and Agresti (1999)) clearly reveal that this behavior can produce erroneous inferences." -- Dorazio and Royle 2003

In a limited set of situations, posterior predictive checks and DIC can identify when non-mixture or exponential TTDDs inadequately describe the marginal pattern of detections over time.  
But in many situations, neither tool points a clear way to select the correct family of TTDD.  
Given the bias that can result from assuming constant detection or omitting a mixture component, conservatism dictates that non-constant TTDDs with a mixture should be preferred.  
Among the TTDDs we considered, the mixture gamma TTDD appears to provide the best combination of accuracy and precision for nonpeaked data.  
For peaked data, gamma and Weibull models seem about equally as good.  
MCMC sampling from gamma models required from 5-10 the run time as lognormal and Weibull models.

If the estimation of effect parameters is the primary interest, then our results suggest that the exact choice of TTDD may not be important.  
Abundance effect estimates are similar regardless of the chosen TTDD, while detection effect estimates are similar for all mixture non-exponential TTDDs.  
These finding may well not hold if the same covariate is modeled in both abundance and detection models \citep{Kery2008}.

Because our focus is detection, we assumed Poisson-distributed abundance for simplicity.  
Abundance distributions that have been used to account for overdispersion include negative binomial, zero-inflated Poisson, and Poisson with survey-level random effects.  
The pattern of counts in our Ovenbird data actually reflect underdispersion, so a Conway-Maxwell Poisson distribution, which can model both overdispersion and underdispersion, may be more appropriate \citep{Wu2015}\adam{apparently, Sellers et al. (in Wu) has a comprehensive overview of the CMP.}

% Overview: Wu15
% ZIP: Etterson, Solymos12
% NB: Royle04(?)


At present, Stan does not sample discrete parameters, so the ability to marginalize out the discrete-valued latent variables $N_{s}$, as we did, can greatly facilitate MCMC sampling.

% \adam{I have contemplated on and off including a discussion of broad priors and the limit behavior as $\varphi \to 0$, but I think it's not really on topic.  
% Note: now that I've seen van Dishoeck and Manntyniemi, perhaps I should worry about informativeness in priors.}

% All estimates are based on extrapolation of the observed detections\\
% -- a) explore the tail more?  I think that'd require a lot of data\\
% -- b) requires extended observation periods or full detection data (Alldredge?)\\
% same goes for sample size

% Really, I have totally ignored the discussion of the true value of p, which I have tended to simulate in a fairly high range


To improve our choice of TTDD and/or its parameters, it makes sense to obtain more data in one of a few ways.  
One approach is to collect complete detection history records, not just first time-to-detection observations \citep{Alldredge2007}.  
This may not be feasible in studies like MNFB, where many species are observed.  
Another approach is to conduct a longer survey that better assesses the tail distribution of the TTDD; however, the longer the survey is, the greater the risk that individuals enter/depart the study area or are double-counted, which violates removal sampling assumptions \citep{LeeMarsden2008, Reidy2011}.  
A third idea is to obtain more precise time-to-detection data, even exact time-to-detection data.  
In some early explorations of our model considering just an exponential mixture TTDD, we found that credible intervals for Ovenbird detection probability were 20-25\% narrower using 9-interval surveys compared to more traditional 3-interval surveys (0-3, 3-5, and 5-10 minutes).  
For all of the above, a small pilot study may be a practical starting point.  


With the increasing use of microphone arrays for automated detection \adam{refs}, it is tempting to think TTDD models may be valuable for microphone-collected data.  
This should be the case when variations in detection result solely from organismal behaviors, such as bout singing, or if we wish to model individual random effects in the detection model.  
However, some explanations for non-constant detection are explicitly related to the observation process --- either effects of the observer on animal behavior or variations in observer effort caused by survey methodology.  
The assumption of constant detection seems more plausible in studies where observers are absent, though detection heterogeneity may still exist across subgroups of the population.

Application of TTDD modeling within the N-mixture framework should be straight-forward for other time-to-detection datasets, such as full detection-history and double-observer.  
The key difference will be in the data model, in that responses will take a different form than multinomially distributed first times to detection.  
But the underlying abundance and detection models remain the same.  
\adam{Time-heterogeneous models have been done for mark-recapture (Farcomeni and Scacciatelli 2013, who reference Hwang and Chao 2002).}

The time-to-detection approach in our model is a departure from the probability-of-detection approach that has been used in distance-removal modeling \citep{Farnsworth2005, Diefenbach2007, Solymos2013, Amundson2014}\adam{Alldredge2007?}.  
In a distance-sampling context, the probability of detection for any individual is framed as the joint probability of two independent detection events: (i) availability for detection, which occurs during an observation period with probability $P_a$, and (ii) detection of the available individual by an observer, $P_b$, also known as perceptibility \citep{Williams2002, Kery2008, Nichols2009}\adam{I haven't read Williams2002 -- cited in Royle2004.  
I'm  beginning to think this may have too-numerous antecedents to cite... in Reidy, the earliest reference is Marsh and Sinclair 1989.  
Ditto Amundson.
See also McCallum 2005.}.  
In avian point-counts, because the overwhelming majority of detections are auditory, $P_a$ is understood to be the probability that a bird vocalizes during the observation period.  
Meanwhile, perceptibility is treated as a function of distance $d$: $P_b = g(d)$, with nearby available individuals being detected with higher probability than distant individuals.  
The probability of detection is thus formulated as availability times perceptibility: $\pdet = P_aP_b$.  
Multiple studies have demonstrated that failure to incorporate distance leads to systemic bias in estimates of abundance \citep{EffordDawson2009, Solymos2013}.

The time-to-detection model as presented here in our study does not incorporate distance.  
Consequently, our definition of a TTDD necessarily represents an averaging across distance classes and a consequent loss of information if distance data are available.  
We can modify our model to account for distance.  
To be consistent with earlier distance-removal studies \citep{Farnsworth2005, Amundson2014}, we could: (i) define a time-to-\textit{availability} distribution $F_T^A(t)$ in the same manner that we have defined a TTDD in this study, (ii) modify the equation for detection probability to reflect the detection distance to individual $q$: $p_{sq}^{(det)} = g(d_q) F_T^A(C_I|\varphi_s, \boldsymbol{\theta})$, and (iii) tweak our data model accordingly.

However, our analysis makes possible a different strategy that is more consistent with a time-to-event conceptualization of the problem.  
Rather than define availability and perceptibility as independent detection events over the course of the observation period, we can define them at the event level.  
In terms of the model we have presented in this paper, this is equivalent to redefining the time-to-detection hazard function, $h_{sq}(t)$ --- which is now a function of the distance to individual $q$ --- as the product of the time-to-availability hazard function $h_{s}^A(t)$ and the perceptibility function $g(d_q)$.  
So, $h_{sq}(t) = g(d_q) h_{s}^A(t)$.  
From standard survival analysis results, $p_{sq}^{(det)} = 1 - \exp\left(-g(d_q)\int\limits_0^{C_I} h_{s}^A(u)du\right) = 1 - \left[1 - F_T^A(C_I|\varphi_s, \boldsymbol{\theta}) \right]^{g(d_q)}$.

In short, the time-to-event conceptualization frames perceptibility as a rate-of-availability multiplier rather than as a probability-of-availability multiplier.  
In a context where rates of availability are allowed to vary from individual to individual, this makes an intuitive sense --- a bird that sings twenty times during the observation period should have its overall detection probability discounted much less than a bird at the same distance which only sings once.  
This is an area of ongoing exploration.


% Skeptics of removal sampling: Johnson08, EffordDawson09, Reidy11

%\subsection{Lit Notes on Distance and Poisson Abundance}
%--- Borchers is all about this\\
%--- Farnsworth2005, EffordDawson2009, Amundson2014\\
%----- particularly noticeable with finite mixture models\\
%- Oedekoven2013 is all about covariates on distance sampling curve\\
%- necessity of using distance, too, is backed by Solymos2013, (Buckland et al 2001, 2004), Farnsworth2005, EffordDawson2009...\\
%
%Poisson assumption (do I want to discuss underdispersion?\\
%--- Denes et al 2015 discuss how Poisson assumption leads to bias in the absence of detection heterog (ref to Solymos et al 2012) or when inflated zeros are present (Joseph et al 2009) [with the obvious remedies being: NegBin and ZIP, respectively]... but is that negated by our underdispersion?\\
%
%`Unmodeled individual heterogeneity causes population size to be underestimated using capture-recapture and removal methods' (from Efford and Dawson, who cite Otis et al. 1978)\\


% Ideas for the future
% (i) Distance-removal time-to-event model
% (ii) Fit more species from this dataset over full survey duration; maybe some model selection (even Bayesian)
% (iii) Take a stab at modeling habitat
% (iv) Model for non-independent singing?
% (v) Model for density-dependent singing? (only valid, I think, if we have lots of underdispersion from(ii))
% (vi) I know I've  been warned off it, but automated detection intrigues me
% (vii) Further testing of existing model: sensitivity to priors, performance for varying detectability levels, sample size, # intervals
% (viii) Incorporate spatial random effects.  
This work has already been done, just not for TTDD models, but the real spatial dependency should be in abundance, not detection
% (ix) Comparison of methods -- just looking at my lit review, there are gadzooks strategies that people have proposed.  
Combining/comparing them seems maybe informative?  To what degree has this already been done?



\begin{appendix}
\section{Supplement}

\begin{table}[!ht]
\centering
\begin{tabular}{lll|ccc}
  \hline
Family & Mixture & Peaked & $\gamma$ & $\varphi$ & $\alpha$ \\ 
  \hline
Exponential & Non-mixture & &  & -1.827 &  \\ 
  & Mixture & & 0.65 & -2.138 &  \\ 
  \hline
  Gamma & Non-mixture & Non-peaked &  & -3.279 & 0.257 \\ 
  & & Peaked &  & -0.746 & 3.371 \\ 
  & Mixture & Non-peaked & 0.65 & -2.773 & 0.577 \\ 
  & & Peaked & 0.65 & -1.210 & 2.491 \\ 
  \hline
  Lognormal & Non-mixture & Non-peaked &  & -0.341 & 2.330 \\ 
  & & Peaked &  & -1.872 & 0.512 \\ 
  & Mixture & Non-peaked & 0.65 & -1.462 & 1.674 \\ 
  & & Peaked & 0.65 & -1.992 & 0.618 \\ 
  \hline
  Weibull & Non-mixture & Non-peaked &  & -1.165 & 0.418 \\ 
  & & Peaked &  & -2.042 & 1.829 \\ 
  & Mixture & Non-peaked & 0.65 & -2.063 & 0.687 \\ 
  & & Peaked & 0.65 & -2.201 & 1.621 \\ 
   \hline
\end{tabular}
\caption{Table of parameter values used to generate data for: (i) the mixture vs. non-mixture simulation, and (ii) the constant vs. non-constant rate simulation.  Here $\gamma$ is a mixing parameter for the proportion of `hard to detect' individuals, $\varphi$ is the detection rate parameter, and $\alpha$ is the shape parameter.}
\end{table}

\begin{table}[ht]
\centering
\begin{tabular}{lllrrrrrrrrrrrrrrrr}
  \hline
Family & Mixture & Peak & Intercept$^A$ & $\beta^A_1$ & $\beta^A_2$ & $\beta^A_3$ & $\beta^A_4$ & Intercept$^D$ & $\beta^D_1$ & $\beta^D_2$ & $\beta^D_3$ & $\beta^D_4$ & $\beta^D_5$ & $\sigma_A[1]$ & $\sigma_A[2]$ & $\sigma_D$ & $\gamma$ & $\alpha$ \\ 
  \hline
Exponential & Non-mixture &  & 0.975 & 0.122 & 0.002 & 0.094 & -1.039 & -1.146 & -0.010 & -0.056 & 0.118 & 0.249 & 0.164 & 0.139 & 0.094 & 0.242 &  &  \\ 
   & Mixture &  & 1.097 & 0.124 & 0.012 & 0.064 & -1.070 & -1.970 & -0.068 & -0.146 & 0.030 & 0.312 & 0.270 & 0.130 & 0.091 & 0.259 & 0.647 &  \\ 
  Gamma & Non-mixture & Non-peaked & 1.383 & 0.129 & -0.001 & 0.098 & -1.039 & -3.941 & 0.048 & -0.104 & -0.120 & 0.141 & 0.129 & 0.121 & 0.072 & 0.390 &  & 0.316 \\ 
   &  & Peaked & 1.134 & 0.129 & -0.001 & 0.098 & -1.039 & -0.746 & 0.048 & -0.104 & -0.120 & 0.141 & 0.129 & 0.121 & 0.072 & 0.390 &  & 3.371 \\ 
   & Mixture & Non-peaked & 1.215 & 0.128 & -0.005 & 0.079 & -1.051 & -2.700 & -0.048 & -0.126 & -0.042 & 0.210 & 0.200 & 0.121 & 0.079 & 0.332 & 0.790 & 0.597 \\ 
   &  & Peaked & 1.134 & 0.128 & -0.005 & 0.079 & -1.051 & -1.210 & -0.048 & -0.126 & -0.042 & 0.210 & 0.200 & 0.121 & 0.079 & 0.332 & 0.650 & 2.491 \\ 
  Lognormal & Non-mixture & Non-peaked & 1.613 & 0.126 & 0.031 & 0.096 & -1.047 & -2.444 & 0.163 & -0.099 & -0.071 & 0.310 & 0.079 & 0.151 & 0.077 & 0.353 &  & 3.287 \\ 
   &  & Peaked & 1.134 & 0.126 & 0.031 & 0.096 & -1.047 & -1.872 & 0.163 & -0.099 & -0.071 & 0.310 & 0.079 & 0.151 & 0.077 & 0.353 &  & 0.512 \\ 
   & Mixture & Non-peaked & 1.282 & 0.126 & 0.017 & 0.070 & -1.068 & -2.051 & 0.006 & -0.140 & -0.038 & 0.341 & 0.191 & 0.140 & 0.079 & 0.312 & 0.706 & 1.467 \\ 
   &  & Peaked & 1.134 & 0.126 & 0.017 & 0.070 & -1.068 & -1.992 & 0.006 & -0.140 & -0.038 & 0.341 & 0.191 & 0.140 & 0.079 & 0.312 & 0.650 & 0.618 \\ 
  Weibull & Non-mixture & Non-peaked & 1.570 & 0.129 & 0.011 & 0.102 & -1.040 & -3.137 & 0.101 & -0.104 & -0.127 & 0.204 & 0.086 & 0.131 & 0.074 & 0.383 &  & 0.387 \\ 
   &  & Peaked & 1.134 & 0.129 & 0.011 & 0.102 & -1.040 & -2.042 & 0.101 & -0.104 & -0.127 & 0.204 & 0.086 & 0.131 & 0.074 & 0.383 &  & 1.829 \\ 
   & Mixture & Non-peaked & 1.294 & 0.129 & -0.002 & 0.081 & -1.049 & -2.354 & -0.028 & -0.122 & -0.069 & 0.204 & 0.183 & 0.122 & 0.078 & 0.344 & 0.814 & 0.653 \\ 
   &  & Peaked & 1.134 & 0.129 & -0.002 & 0.081 & -1.049 & -2.201 & -0.028 & -0.122 & -0.069 & 0.204 & 0.183 & 0.122 & 0.078 & 0.344 & 0.650 & 1.621 \\ 
   \hline
\end{tabular}
\caption{\label{tbl:sim3}Table of parameter values used to generate data for simulations with covariates.  Here, $\gamma$ is a mixing parameter for the proportion of `hard to detect' individuals, $\beta$'s are fixed effects, $\sigma$'s are random effect standard deviations, $\alpha$ is a shape parameter, and superscripts of $A$ and $D$ indicate abundance and detection parameters, respectively.}
\end{table}

\begin{figure}[h!]\centering
\includegraphics[width=0.98\textwidth]{Sims/SimFull/pdet_cater_correct.pdf}
\caption{\label{pdet_cater_correct} Sim3 caterpillar plots of posterior 50\% and 95\% credible intervals for the marginal probability of detection $\pdet$.  
Inference models come from the same family as the dataset but may differ in the presence/absence of a mixture component.  
Each column presents one family of simulated dataset.  
Upper plots show non-mixture datasets; lower plots show mixture datasets.  
`X' marks the expected marginal probability of detection based on true parameter values.}
\includegraphics[width=0.98\textwidth]{Sims/SimFull/pdet_cater_family.pdf}
\caption{\label{pdet_cater_family}  Sim3 caterpillar plots of posterior 50\% and 95\% credible intervals for the marginal probability of detection $\pdet$. All data and inference models include mixtures.  
Each plot presents one simulated dataset.  
`X' marks the expected marginal probability of detection based on true parameter values.}
\end{figure}

\end{appendix}

\bibliography{masterbib}
\bibliographystyle{biom}

\end{document}
